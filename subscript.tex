\documentclass{article}
\usepackage[UTF8]{ctex}
\usepackage{hyperref}

\linespread{1.3}
\setlength{\parskip}{0.5em}
\setlength{\parindent}{0em}

\begin{document}

\section*{Quick Thinks: AI Has Entered the Chat — a “Conversation” with ChatGPT}

In this podcast episode, Matt Abrahams interviews an algorithm.

\begin{flushright}
    \textit{February 08, 2023 | by Matt Abrahams}
\end{flushright}

In this \textsl{Quick Thinks} episode, podcast host Matt Abrahams interviews ChatGPT, a natural language processing chatbot, about its purpose, sources of information, and ethical considerations.

\textsl{\textbf{Editor’s Note:} To produce this podcast, Matt typed his questions into \href{https://openai.com/blog/chatgpt/" target="_blank}{ChatGPT} to create the conversation, then recorded an audio version of those questions in the studio. Our producers then used a text-to-speech tool to give voice to ChatGPT.}

\vspace{1em}
\large{\textbf{TRANSCRIPT}}
\vspace{1em}



\textbf{Matt Abrahams: }In my career, I have seen several technological advances that have changed the way we communicate, but none have shaken the world as much as AI has recently.

I’m Matt Abrahams and I teach Strategic Communication at Stanford Graduate School of Business. Welcome to this Quick Thinks episode of Think Fast Talk Smart, the podcast.

Today I am excited to run an experiment. For the first time ever on this podcast, I will not be interviewing a person. Instead, I’ll be interviewing an algorithm: ChatGPT.

ChatGPT is an natural language processing chatbot that allows users to have human-like conversations. The G. P. T. Stands for generative pre-trained transformer, and it was launched by OpenAI in November of 2022. For today’s experiment and interview, we will have typed in the answers and we will be using a programmed language to read out chat GPTs responses. So let’s get started with our experiment.

Welcome ChatGPT. Thanks for being here.

\textbf{Chat GPT: }You’re welcome! I’m here to help, so feel free to ask me anything.

\textbf{Matt Abrahams: }Great, let’s get started. For those of our listeners who aren’t familiar with exactly what you are, can you define your purpose and, and give us two examples.

\textbf{Chat GPT: }The purpose of ChatGPT is to provide informative and accurate responses to questions in natural language. Two examples: answering factual questions, for example: “What is the capital of France?” And, generating creative text, for example: “Write a story about a dragon.”

\textbf{Matt Abrahams: }Wow. Now that’s impressive. Where do you, ChatGPT, get the information you use to answer your questions?

\textbf{Chat GPT: }ChatGPT is a language model created by Open AI that has been trained on vast amounts of text data from the internet. It uses this training to generate responses to questions based on patterns it has seen in the data.

\textbf{Matt Abrahams: }Essentially, you know what you know by pulling from all the data on the internet. So in some ways, you get your answers the way we do, through reflection and learning and training to help us generate our responses. Right?

\textbf{Chat GPT: }Yes, that’s correct. ChatGPT has been trained on vast amounts of text data from the internet, allowing it to generate responses to questions based on patterns and information found in that data. However, it’s important to note that not all information on the internet is accurate or up-to-date, so the responses generated by ChatGPT may not always be 100% correct. 

\textbf{Matt Abrahams: }Ah, so now you bring me directly to my next question, which is what safeguards does ChatGPT use to ensure its answers are appropriate, accurate, and ethical?

\textbf{Chat GPT: }It is important to critically evaluate the responses generated by ChatGPT and not rely on them as the sole source of information.

\textbf{Matt Abrahams: }I see. Hmm. Now many teachers like myself, hiring managers and college admissions officers worry that ChatGPT might lead people to cheat, to be disingenuous. What do you think about that?

\textbf{Chat GPT: }Cheating undermines the learning and evaluation process and is not to be encouraged or condoned. While ChatGPT can generate responses to questions, it is important for individuals to understand the importance of original work and to not use the technology to substitute their own efforts. The ultimate responsibility for ensuring academic integrity and ethical behavior lies with the individuals using Chat GPT, and they should use it in a manner that aligns with their institution’s policies and guidelines.

\textbf{Matt Abrahams: }So it’s the user’s responsibility to use the tool appropriately. Now, let me ask you for some advice and guidance in a fun way. How would you suggest someone reduce their speaking anxiety? But I want you to give your advice as a Dr. Seuss rhyme.

\textbf{Chat GPT: }When it comes to speaking, do not be so blue.<br/>
Just a few tips to help get you through.<br/>
Take a deep breath, it’ll calm your mind.<br/>
And think of what you want to say. Be kind.<br/>
Remember, it’s normal to feel a bit of fear.<br/>
But pushing through will bring you cheer.<br/>
So be confident, and speak with pride.<br/>
Your words will shine, they’ll surely be a guide!

\textbf{Matt Abrahams: }Hey, that’s not bad at all. Uh, it’s good advice and certainly a fun way to hear it. Do we still need to learn how to communicate better now that you exist?

\textbf{Chat GPT: }Yes, effective communication skills are still essential, even with the existence of ChatGPT. While ChatGPT can provide information and generate responses, it cannot replace the human interaction and emotional intelligence that comes with face-to-face communication. Furthermore, not all communication can be done through technology, and the ability to communicate effectively in person remains important in many settings such as business meetings, personal relationships, and public speaking. Additionally, developing strong communication skills can enhance one’s confidence and ability to express themselves effectively.

\textbf{Matt Abrahams: }Well, that’s good to know. Before we end, I’d like to ask you the same three questions I ask everyone who joins me.

Question one. If you were to capture the best communication advice you ever received as a five-to-seven word presentation slide title, what would it be?

\textbf{Chat GPT: }Listen, Speak Authentically, Show Empathy.

\textbf{Matt Abrahams: }Wow. Your response is very similar to what we’ve heard from many of our guests. You have to listen to really know what’s needed in the moment. You wanna speak your truth, be authentic to yourself, and you absolutely want to demonstrate empathy and concern and focus for your audience. Let me ask question number two. Who is a communicator that you admire and why?

\textbf{Chat GPT: }One communicator I admire is Martin Luther King, Jr.. His speeches were powerful, impactful, and delivered with passionate conviction. He effectively communicated his message of equality and justice for all, inspiring social change through his words. His use of metaphors, repetition, and vivid imagery made his speeches memorable and enduring. He remains a powerful example of the impact that effective communication can have.

\textbf{Matt Abrahams: }Well, absolutely. Martin Luther King Jr. Is an amazing example of a good speaker. All of us are familiar with his I Have a Dream speech. I actually find his Nobel Prize acceptance speech to be on par, if not even better, than the “I Have a Dream” speech. Thanks for sharing.

Question three. What are the first three ingredients that go into a successful communication recipe?

\textbf{Chat GPT: }Active Listening, Clarity, and Empathy.

Well, your response is, no surprise, it lines up with the slide title that you shared before. I find it fascinating that an AI algorithm is talking about listening and empathy. I have to spend some time thinking about that.

\textbf{Matt Abrahams: }Well, ChatGPT. Thank you for being here. Thank you for sharing your thoughts.

I find this really intriguing and interesting, and I can truly see the value of a tool like yourself in helping us to prepare to take on and create different options for our communication. Clearly, if I were a non-native speaker, It could be very helpful to get ideas about how to formulate these thoughts and use the proper vocabulary, but I have to admit, I have some concerns.

I’m not sure about the emotion, the connection, the validity of some of the information. Like many of us, I am intrigued by, but also concerned about what you bring to our world of communication. Thank you for being here and for making us think.

\textbf{Chat GPT: }You’re welcome! I’m here to help.

\textbf{Matt Abrahams: }Okay. ChatGPT, one last favor. Can you write our outro please?

\textbf{Chat GPT: }Yes I can.

That’s it for today’s episode of \textsl{Think Fast, Talk Smart}. We hope you found it informative and thought-provoking. If you enjoyed the show, please share it with your friends and colleagues. We’d also love it if you subscribed to the podcast and followed us on LinkedIn and Instagram. And don’t forget to sign up for our email newsletter to stay up-to-date on all things related to communication and public speaking. Thanks for listening and until next time, keep thinking fast and talking smart.



\end{document}
